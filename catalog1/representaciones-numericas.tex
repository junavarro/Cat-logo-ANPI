\section {Representaciones numéricas}
\subsection{Números enteros sin signo de N bits}
 $$
  x = \sum_{n=0}^{N-1} b_n2^{n}
 $$ Donde  $b_n \in \{ 0,1 \} $ y n es el n-ésimo digito de x, el rango representable es de  0 hasta $ 2^N -1 $
\subsection{Número coma fija sin signo de N bits}
Donde  $b_n \in \{ 0,1 \} $ y n es el n-ésimo digito de x, M es una constante de normalización, usualmente igual a $ 2^m $
$$
	 x = \frac{1}{M = 2^m}\sum_{n=0}^{N-1} b_n2^{n}	
$$ Donde  m = es el número de Bits de la parte fraccionaria. Por tanto se representa el número como si no fuara fraccionario y luego se divide entre M.

\subsection{Numero entero con signo  de N bits en complemento a dos : }
$$
 x = -b_{N-1}2^{N-1} +  \sum_{n =0}^{N-2} b_n2^n
$$ Lo que permite representar números de $ -2^{N-1}$ hasta $ 2^{N-1} -1  $ El signo está definido por el bit más significativo.
\subsection{ Coma fija con signo }
$$
 x = \frac{1}{M} = \big( -b_{N-1}2^{N-1} +  \sum_{n =0}^{N-2} b_n2^n  \big) 
$$
\subsection {Alternativa  Caso fraccionario}
$$
  x = -b_{N-1}+ \sum_{n=0}^{N-2} b_n2^{n-N+1}
$$
\subsection{ Coma flotante}
\begin{itemize}
\item Bit de signo s
\item Exponente e con E bits
\item Mantisa Normalizada (fraccionaria) de M bits
\end{itemize}
$$
 x = \big(  -1 \big)^s X \big(1,\big) X 2^{e - bias}
$$ En donde  bias está definido como: $ bias = 2^{E-1} -1 $
\subsection{ Epsilon de Formato}
$$
 \epsilon = >= \frac{|\triangle x|}{|x| } 
$$
En caso de que use redondeo $\frac {\epsilon}{2} = >= \frac{|\triangle x|}{|x| } $